\chapter{Introducción}\label{cap.introduccion}
\pagenumbering{arabic}

Este capítulo presenta una introducción y panorama general de la tesis de magíster, la cual propone un nuevo tipo de actas de reuniones basada en una teoría del diálogo llamada diálogo/acción. Primero, se da una descripción del problema propuesto después la respuesta o solución a esa problemática. A continuación, se muestran los alcances del modelo de desarrollo para lograr la un solución de software. Finalmente, se describe brevemente el contenido de los siguientes capítulos.


\section{ANTECEDENTES Y MOTIVACIÓN}

El área del CSCW\footnote{Para ver la definición de \textit{Computer Supported Cooperative Work} en este enlace \url{https://en.wikipedia.org/wiki/Computer-supported_cooperative_work}} es un campo multidisciplinario de investigación sobre el fenómeno de la colaboración y su relación con la tecnología informática. En ese campo se han desarrollado tecnologías como: sistema de videoconferencia, pizarras compartidas, sistema de intercambio, flujos de trabajo (en Inglés, \textit{workflow}); todas con enfoque de apoyo colaborativo al trabajo en grupo \cite{RN32}. A su vez, se ha han desarrollado marcos teóricos sobre actividades de trabajo en equipo como la presentada en \cite{RN24} que propone un enfoque llamado diálogo/acción\footnote{El diálogo/acción es una extensión del enfoque de lenguaje/acción propuesto originalmente por Terry Winograd y Fernando Flores (1986) y presentado en su libro “Understanding Computers and Cognition”. La diferencia principal entre ambos enfoques es que en lenguaje/acción centra exclusivamente en la parte estable de las conversaciones, es decir en los compromisos y los acuerdos. En cambio, en el enfoque diálogo/acción incorpora la parte divergente de la comunicación constituida por las dudas y los desacuerdos.} con sus artefactos tecnológicos asociados pero cuya efectividad no se demuestra aún.

Las teorías y modelos más recientes permiten comprender mejor los procesos de cooperación \cite{RN27}, con el objeto de dar efectividad a las reuniones de trabajo. Sin embargo, ello debe complementarse con nuevas herramientas CSCW (llamadas groupware) más específicamente del área  \textit{meetingware} - para situar a las personas (mismo tiempo) en cualquier hilo conversacional (mismo lugar) de cualquier proyecto colectivo. Esto se infiere de acuerdo a la taxonomía de clasificación no excluyente \cite{RN2}, debido a que el proyecto podría emplearse para funciones diferentes a las que fue creado, esto de acuerdo al siguiente cuadro:

\begin{table}[]
\caption{ taxonomía de clasificación no excluyente para D-Minute, adaptado de  (Penichet, Marin, Gallud, Lozano,  Tesoriero, 2007)}
\label{my-label}
\begin{tabular}{@{}lll@{}}
\toprule
Herramienta & \multicolumn{2}{l}{D-Minute} \\ \midrule
\multicolumn{1}{|c|}{\multirow{3}{*}{Característica CSCW}} & \multicolumn{1}{l|}{Colabora} & \multicolumn{1}{l|}{1} \\ \cmidrule(l){2-3} 
\multicolumn{1}{|c|}{} & \multicolumn{1}{l|}{Comunica} & \multicolumn{1}{l|}{0} \\ \cmidrule(l){2-3} 
\multicolumn{1}{|c|}{} & \multicolumn{1}{l|}{Coordina} & \multicolumn{1}{l|}{1} \\ \midrule
\multicolumn{1}{|l|}{\multirow{2}{*}{Tiempo}} & \multicolumn{1}{l|}{Sincrono} & \multicolumn{1}{l|}{1} \\ \cmidrule(l){2-3} 
\multicolumn{1}{|l|}{} & \multicolumn{1}{l|}{Asincrono} & \multicolumn{1}{l|}{0} \\ \midrule
\multicolumn{1}{|l|}{\multirow{2}{*}{Espacio}} & \multicolumn{1}{l|}{Mismo Lugar} & \multicolumn{1}{l|}{1} \\ \cmidrule(l){2-3} 
\multicolumn{1}{|l|}{} & Diferente Lugar & 0 \\ \bottomrule
\end{tabular}
\end{table}