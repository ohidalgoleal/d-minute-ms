%\documentclass[10pt]{book}%
\documentclass[a4paper,openright,12pt]{book}


%PACKAGE%
\usepackage[latin1]{inputenc}
\usepackage{blindtext}
\usepackage{geometry}
\usepackage{fancyhdr}
\usepackage{helvet}
\usepackage[spanish,es-tabla]{babel}
\usepackage[breaklinks=true]{hyperref}
\usepackage{multirow}
\usepackage{apacite}
\usepackage{booktabs}
\usepackage[utf8]{inputenc} 

%CONFIGURACIONES%
\setcounter{secnumdepth}{3} %para que ponga 1.1.1.1 en subsubsecciones
\setcounter{tocdepth}{3} % para que ponga subsubsecciones en el indice
\renewcommand{\baselinestretch}{1,5} %Interlineado
\pagestyle{fancy}
% Clear the header and footer
\fancyhead{}
\fancyfoot{}
\renewcommand{\familydefault}{\sfdefault} 
\makeatletter
% we use \prefix@<level> only if it is defined
\renewcommand{\@seccntformat}[1]{%
  \ifcsname prefix@#1\endcsname
    \csname prefix@#1\endcsname
  \else
    \csname the#1\endcsname\quad
  \fi}
% define \prefix@section
\newcommand\prefix@section{CAPITULO \thesection. }
\makeatother


%DOCUMENTO%
\begin{document}

\begin{titlepage}
\begin{center}
\begin{Huge}
\textsc{Un buen título es muy importante}
\end{Huge}
\end{center}
\end{titlepage}

% para crear una cara en blanco
\newpage
$\ $
\thispagestyle{empty} % para que no se numere esta página

\chapter*{}
% Set the right side of the footer to be the page number
\fancyfoot[R]{\thepage}
\pagenumbering{Roman} % para comenzar la numeración de paginas en números romanos
\begin{flushright}
\textit{Dedicado a \\
mi familia}
\end{flushright}

\chapter*{Resumen} % si no queremos que añada la palabra "Capitulo"
\addcontentsline{toc}{chapter}{Resumen} % si queremos que aparezca en el índice
El objetivo principal de la tesis es generar un tipo de actas de reuniones que facilite recuperar el estado de reuniones pasadas. Para lograr esto se presenta - en el mismo documento - la creación de un software denominado D-Minute, que incorpora la teoría del diálogo al momento de la confección de un acta de reuniones donde los participantes interactúan en el mismo espacio de manera síncrona. El tema se justifica por el problema de las altas horas improductivas que se pasan en reuniones con casi ningún marco conceptual para enfrentar este gran dilema que es ubicuo en el mundo. Luego,  se busca un software donde cada participante de la reunión genere la colaboración y coordinación de actividades mediante los elementos de dialogo: duda, desacuerdo, compromiso, norma, acuerdo y compromiso individual. El documento tiene sus cimientos en la teoría del diálogo/acción, donde el diálogo se entiende como una forma de comunicación que facilita la creación y este caso un meetingware que apoya la gestión haciendo uso elementos de comunicación articulado con la acción.
Este trabajo también dio lugar a una investigación de innovación, donde se analizaron diferentes herramientas del mercado que permiten generar actas de reuniones de las cuales muy pocas utilizan elementos del diálogo para el seguimiento de tareas. Lo anterior permitió generar un modelo liviano de negocio mediante metodología Lean Startup.
Una parte relevante de la tesis fue la utilización de Scrum como metodología de desarrollo de software, pasando las épicas a su correspondiente release map y las historias de usuario asociadas. La segunda parte relevante fue el desarrollo en microservicios utilizando la arquitectura de Netflix OSS combinado con angular 5 en la parte frontal.
Por último, se realiza un benchmarking en base a los criterios que debe poseer una herramienta de gestión de reuniones, los criterios fueron tomados de la revisión literatura y del análisis de los sistemas de mercado. El resultado final es la medición de las características del software D-Minute con sus competidores de mercado y una base para trabajos futuros. \newline 
\newline\textbf{Palabras Clave:} Trabajo Colaborativo, Diálogo, Meetingware

\tableofcontents % indice de contenidos

\cleardoublepage
\addcontentsline{toc}{chapter}{Lista de figuras} % para que aparezca en el indice de contenidos
\listoffigures % indice de figuras

\cleardoublepage
\addcontentsline{toc}{chapter}{Lista de tablas} % para que aparezca en el indice de contenidos
\listoftables % indice de tablas 


\chapter{Introducción}\label{cap.introduccion}
\pagenumbering{arabic}

Este capítulo presenta una introducción y panorama general de la tesis de magíster, la cual propone un nuevo tipo de actas de reuniones basada en una teoría del diálogo llamada diálogo/acción. Primero, se da una descripción del problema propuesto después la respuesta o solución a esa problemática. A continuación, se muestran los alcances del modelo de desarrollo para lograr la un solución de software. Finalmente, se describe brevemente el contenido de los siguientes capítulos.


\section{ANTECEDENTES Y MOTIVACIÓN}

El área del CSCW\footnote{Para ver la definición de \textit{Computer Supported Cooperative Work} en este enlace \url{https://en.wikipedia.org/wiki/Computer-supported_cooperative_work}} es un campo multidisciplinario de investigación sobre el fenómeno de la colaboración y su relación con la tecnología informática. En ese campo se han desarrollado tecnologías como: sistema de videoconferencia, pizarras compartidas, sistema de intercambio, flujos de trabajo (en Inglés, \textit{workflow}); todas con enfoque de apoyo colaborativo al trabajo en grupo \cite{RN32}. A su vez, se ha han desarrollado marcos teóricos sobre actividades de trabajo en equipo como la presentada en \cite{RN24} que propone un enfoque llamado diálogo/acción\footnote{El diálogo/acción es una extensión del enfoque de lenguaje/acción propuesto originalmente por Terry Winograd y Fernando Flores (1986) y presentado en su libro “Understanding Computers and Cognition”. La diferencia principal entre ambos enfoques es que en lenguaje/acción centra exclusivamente en la parte estable de las conversaciones, es decir en los compromisos y los acuerdos. En cambio, en el enfoque diálogo/acción incorpora la parte divergente de la comunicación constituida por las dudas y los desacuerdos.} con sus artefactos tecnológicos asociados pero cuya efectividad no se demuestra aún.

Las teorías y modelos más recientes permiten comprender mejor los procesos de cooperación \cite{RN27}, con el objeto de dar efectividad a las reuniones de trabajo. Sin embargo, ello debe complementarse con nuevas herramientas CSCW (llamadas groupware) más específicamente del área  \textit{meetingware} - para situar a las personas (mismo tiempo) en cualquier hilo conversacional (mismo lugar) de cualquier proyecto colectivo. Esto se infiere de acuerdo a la taxonomía de clasificación no excluyente \cite{RN2}, debido a que el proyecto podría emplearse para funciones diferentes a las que fue creado, esto de acuerdo al siguiente cuadro:

\begin{table}[]
\caption{ taxonomía de clasificación no excluyente para D-Minute, adaptado de  (Penichet, Marin, Gallud, Lozano,  Tesoriero, 2007)}
\label{my-label}
\begin{tabular}{@{}lll@{}}
\toprule
Herramienta & \multicolumn{2}{l}{D-Minute} \\ \midrule
\multicolumn{1}{|c|}{\multirow{3}{*}{Característica CSCW}} & \multicolumn{1}{l|}{Colabora} & \multicolumn{1}{l|}{1} \\ \cmidrule(l){2-3} 
\multicolumn{1}{|c|}{} & \multicolumn{1}{l|}{Comunica} & \multicolumn{1}{l|}{0} \\ \cmidrule(l){2-3} 
\multicolumn{1}{|c|}{} & \multicolumn{1}{l|}{Coordina} & \multicolumn{1}{l|}{1} \\ \midrule
\multicolumn{1}{|l|}{\multirow{2}{*}{Tiempo}} & \multicolumn{1}{l|}{Sincrono} & \multicolumn{1}{l|}{1} \\ \cmidrule(l){2-3} 
\multicolumn{1}{|l|}{} & \multicolumn{1}{l|}{Asincrono} & \multicolumn{1}{l|}{0} \\ \midrule
\multicolumn{1}{|l|}{\multirow{2}{*}{Espacio}} & \multicolumn{1}{l|}{Mismo Lugar} & \multicolumn{1}{l|}{1} \\ \cmidrule(l){2-3} 
\multicolumn{1}{|l|}{} & Diferente Lugar & 0 \\ \bottomrule
\end{tabular}
\end{table}



%----------------------------------------------------------------------------------------
%	BIBLIOGRAPHY
%----------------------------------------------------------------------------------------
\bibliographystyle{apacite}
\bibliography{TesisMagister}
\nocite{*}

%----------------------------------------------------------------------------------------

\end{document}